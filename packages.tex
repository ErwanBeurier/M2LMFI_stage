% Ce fichier contient les packages que j'utilise le plus souvent.

\usepackage{amsmath, amsfonts, amssymb, stmaryrd, mathtools}

\usepackage{shorttoc}
\usepackage{color}
\usepackage[T1]{fontenc}
\usepackage{lmodern}
\usepackage{pict2e}
\usepackage{textcomp}
\usepackage[utf8]{inputenc}
\usepackage{graphicx}
\usepackage[frenchb]{babel}

\usepackage[top=1.5cm, bottom=1.5cm, left=1.5cm, right=1.5cm]{geometry} % Pour contrôler la largeur des marges
%\usepackage[pdftex=true,colorlinks=true,linkcolor=blue]{hyperref} % Ça conflicte avec uasiment toute la suite. 
\usepackage[pdftex=true, hidelinks]{hyperref} % Package pour les hyperliens (et mettre des ancres partout !)
\frenchbsetup{StandardLists = true} % Pour que le package suivant fonctionne bien
\usepackage{enumitem} % Pour les listes imbriquées avec personnalisation de l'item (niveau 1 : chiffres ; niveau 2 : lettres par exemple)
\usepackage[english, onelanguage, boxed]{algorithm2e}
\usepackage{listings}
\usepackage{import}


% Un package sympathique pour donner un contexte aux définitions, théorèmes, etc. 

\usepackage{amsthm}

\newtheoremstyle{smallcapsit}% small caps and italic
{2ex}% above space 
{2ex}% below space 
{\itshape}% body font 
{}% indent amount 
{\scshape}% head font 
{.}% post head punctuation 
{ }% post head punctuation 
{}% head spec
% Met l'intitulé en petites capitales (donc une vraie majuscule pour le titre et des majusucules plus petites pour les autres)

\theoremstyle{smallcapsit}
\newtheorem{theorem}{Théorème}
\newtheorem{prop}{Proposition}
\newtheorem{coro}{Corollaire}
\newtheorem{lemma}{Lemme}
\newtheorem{ax}{Axiome} % Why do I have that ?
\newtheorem*{theorem*}{Théorème}
\newtheorem*{prop*}{Proposition}
\newtheorem*{coro*}{Corollaire}
\newtheorem*{lemma*}{Lemme}
\newtheorem*{ax*}{Axiome}
\newtheorem{conj}{Conjecture}



%\theoremstyle{definition}
\theoremstyle{theorem}
\newtheorem{definition}{Définition}
\newtheorem{example}{Exemple}
\newtheorem{algo}{ALGORITHME}
\newtheorem*{definition*}{Définition}
\newtheorem*{example*}{Exemple}
\newtheorem*{algo*}{ALGORITHME}

\theoremstyle{remark}
\newtheorem{remark}[theorem]{Remarque}
\newtheorem{prob}{Problème}
\newtheorem*{demo}{Démonstration}

