% Ce fichier contient les macros/notations que j'ai définies depuis que j'utilise LaTeX.


% Commandes personnalisées

	% Notations standards

\newcommand{\naturels}{\mathbb{N}}
\newcommand{\reels}{\mathbb{R}}
\newcommand{\complexes}{\mathbb{C}}
\newcommand{\crod}{\right]\,\!\!\!\right]} % Oui, les espaces sont bizarres, mais parfois ça ne fait pas ce que je veux, donc je suis obligé de modifier ça.
\newcommand{\crog}{\left[\,\!\!\!\left[}
\newcommand{\diff}{\mathrm{d}}
\newcommand{\re}[1]{\text{Ré}\left( #1 \right)}
\newcommand{\im}[1]{\text{Im}\left( #1 \right)}
\newcommand{\bb}[1]{\mathbb{#1}}
\newcommand{\bigunion}[2]{\underset{#1}{\bigcup} #2}
\newcommand{\biginter}[2]{\underset{#1}{\bigcap} #2}
% Respectivement symbole d'union et d'intersection sur des ensembles indexés. S'utilise ainsi :
% \bigunion{indices}{ensembles}
\newcommand{\intint}[2]{\crog #1 , #2 \crod }
\newcommand{\minset}[1]{\min\left( \left\lbrace #1 \right\rbrace \right)}
\newcommand{\minlist}[2]{\underset{#2}{\min}\left( #1 \right)}
\newcommand{\maxset}[1]{\max\left( \left\lbrace #1 \right\rbrace \right)}
\newcommand{\maxlist}[2]{\underset{#2}{\max}\left( #1 \right)}
\newcommand{\grozo}[1]{\mathcal{O}\left( #1 \right)}

	% Notations moins standards
	
\newcommand{\premiers}{\mathbb{P}}
\newcommand{\zetaS}{\sum\limits_{n = 1}^{\infty} \frac{1}{n^s}}
\newcommand{\zetaP}{\underset{p \in \mathbb{P}}{\prod} \frac{1}{1- \frac{1}{p^s}}}
\newcommand{\pprod}{\underset{p \in \mathbb{P}}{\prod}}
\newcommand{\eqT}{\underset{T \rightarrow + \infty}{=}}
\newcommand{\Ketten}{\overset{\infty}{\underset{n=1}{\mathrm{\text{\Large{K}}}}}}

\newcommand{\nuplet}[3]{\left(#1_{#2}, \dots, #1_{#3}\right)}
\newcommand{\nupletf}[4]{\left( #1\left(#2_{#3}\right), \dots, #1\left(#2_{#4}\right) \right)}
\newcommand{\card}[1]{\text{card}\left( \left\lbrace #1 \right\rbrace \right)}
\newcommand{\abs}[1]{\left|#1\right|}
\newcommand{\maxs}[1]{\max\left( #1 \right)}
% Définit une commande commode pour définir un n-uplet. La première numérote 
% le premier symbole, la deuxième macro ajoute un symbole (typiquement, une 
% fonction) autour des éléments énumérés.
% Exemple 1 :	\nuplet{a}{1}{n} ==> (a_1, ..., a_n)
% Exemple 2 : 	\nupletf{f}{a}{1}{n} ==> (f(a_1), ..., f(a_n))

	% Tics mathématiques
	
\newcommand{\DCC}{\textit{D.C.C.}} % Dans ces conditions
\newcommand{\casedist}[1]{\left\lbrace  
		\begin{array}{ll}
			#1
		\end{array}
	\right.}


	% Commandes non mathématiques

\newcommand{\espace}{\vspace{3mm}}
\newcommand{\bfunder}[1]{\textbf{\underline{#1}}}
\newcommand{\textquoted}[1]{\textquotedblleft#1\textquotedblright}

	% Couleurs
	
\newcommand{\temp}[1]{\textcolor[RGB]{170,170,170}{ #1 }}
\newcommand{\nepasoublier}[1]{\colorbox{red}{#1}}
\newcommand{\redtext}[1]{\textcolor{red}{#1}}

